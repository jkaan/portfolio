\section{Bewijslast}
Dit hoofdstuk bevat de bewijslast voor de leerdoelen die in het Plan van Aanpak staan. Hier zal een apart hoofdstuk per leerdoel komen waar in diepte wordt uitgelegd waarom dit leerdoel is behaald.

\subsection{Facebook API}

\subsection{Meteor, een JavaScript framework}

\subsection{Versiebeheer}
Het desbetreffende leerdoel is: Versiebeheer volgens de Gitflow workflow gebruiken zodat er altijd een fout vrije en uitrolbare versie van de applicatie beschikbaar is en als er een fout optreedt deze snel weer ongedaan gemaakt kan worden door naar een vorige versie terug te keren.

Ik heb voor Content2Connect een aantal belangrijke fixes moeten doen, deze hadden hoge prioriteit. Deze komen volgens de Gitflow workflow dan op zogenaamde feature branches. Er is geen speciale conventie voor hoe deze feature branches moeten heetten maar heel veel bedrijven houden de volgende conventie aan: feature/[feature-naam], dit heb ik dus ook gedaan. Mijn branch voor deze belangrijke fixes heet daarom ook feature/important-fixes. Bij ConnectSB wordt gewerkt met Redmine en hierbij kan er prioriteit gegeven worden aan bepaalde issues. Deze hoge prioriteit komt dus ook van Redmine.

Voor een grote hoeveelheid met kleine issues is het in mijn belevenis beter om een branch te maken die deze allemaal bevat. Veel bedrijven houden zich ook aan het aanmaken van één branch voor elke issue, maar dit is naar mijn idee geen goed idee. Als een issue natuurlijk best wel uitgebreid is en je waarschijnlijk kleine taken zult hebben is dit een goed idee, maar als je één issue moet doen die hooguit 5 minuten duurt is dit het natuurlijk niet echt waard.

Ik moest in Content2Connect bij het overzicht van één bestelling de titel toevoegen van de bestelling. Hiervoor wordt er dan een branch gemaakt genaamd feature/change-#2318 want de issue heet op Redmine Change #2318. Change is dan de categorie en #2318 is het nummer van de issue.

Dit heb ik dus eigenlijk nu gedaan voor elke issue die er op Redmine staat. Deze branches komen eigenlijk nooit op de centrale repository. Deze kunnen gepusht worden als dat nodig is, bijvoorbeeld als iemand mee moet helpen om aan een bepaalde feature mee te werken. Features zullen meest van de tijd, in ieder geval in mijn situatie, werk bevatten voor één persoon dus de feature branches hoeven dan nooit gepusht te worden.

Ik heb vandaag ook een release uitgebracht van Content2Connect. Dit is versie v0.9, hier wordt dan een tag voor aangemaakt op de master en dan kan deze in productie gezet worden.

Gitflow workflow werkt vooral heel goed als je bijvoorbeeld bezig bent aan nieuwe functionaliteit, maar je baas zegt dat er even snel wat anders moet gebeuren. Je maakt een nieuwe feature branch aan, commit daar alles op. Deze merge je dan met de develop branch en indien nodig kan deze naar master voor een nieuwe release.

\subsection{Functioneel ontwerp}

