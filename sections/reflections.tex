\section{Reflecties}
Tijdens het uitvoeren van het maken van de producten en het vergaren van de bewijslast voor de leerdoelen in het Plan van Aanpak moet er natuurlijk ook gereflecteerd worden op de uitgevoerde werkzaamheden.

\noindent

\begin{tabu} to \textwidth { | l | X[l] | }
\hline
\multicolumn{2}{|l|}{Academie: } \\ 
\hline
\multicolumn{2}{|l|}{Opleiding: } \\ 
\hline
\multicolumn{2}{|l|}{Studentnaam: Joey Kaan \hspace{35pt} Studentnummer: 64808} \\
\hline
\multicolumn{2}{|l|}{Stagedocent: D. de Waard} \\ 
\hline
\multicolumn{2}{|l|}{Leerdoel/competentie: } \\
\hline
\multicolumn{2}{|l|}{Cursus: \hspace{35pt} Cursuscode: } \\
\hline
\multicolumn{2}{|l|}{Datum: \today} \\
\hline
\multicolumn{2}{|l|}{Titel en nummer van bewijs/bewijzen: } \\ [50pt]
\hline
\multicolumn{2}{|l|}{Oordeel: } \\
\hline
S & Geef voorbeelden van opdrachten (situaties) waarmee je kunt aantonen dat je de competentie hebt verworven. Beschrijf kort wat er aan de hand was of om welke opdracht het ging. \\ [60pt]
\hline
T & Beschrijf de exacte rol/taak die jij had. Geef aan of het om een complexe taak ging en waaruit dat bleek. Wat moest jij doen? \\ [60pt]
\hline
A & Beschrijf de activiteiten die jij achtereenvolgens hebt ondernomen in het kader van deze opdracht. Wat heb je concreet gedaan? \\ [60pt]
\hline
R & Beschrijf het resultaat van de opdracht en hoe de betrokken en er op reageerden. Wat is er vervolgens met dat resultaat gebeurd? \\ [60pt]
\hline
R & Wat heb je ervan geleerd? Wat zou je volgende keer anders aanpakken en waarom? \\ [60pt]
\hline
T & Geef een voorbeeld van een andere situatie waarin je deze competentie kunt toepassen \\ [60pt]
\hline
\end{tabu}