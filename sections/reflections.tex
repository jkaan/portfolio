\section{Reflecties}
Tijdens het uitvoeren van het maken van de producten en het vergaren van de bewijslast voor de leerdoelen in het Plan van Aanpak moet er natuurlijk ook gereflecteerd worden op de uitgevoerde werkzaamheden.

\begin{tabularx}{\textwidth}{| l | X |}
\hline
\rowcolor{yellow}
\multicolumn{2}{|l|}{Academie: Academie voor Technologie \& Innovatie } \\ 
\hline
\rowcolor{yellow}
\multicolumn{2}{|l|}{Opleiding: HBO-ICT } \\ 
\hline
\rowcolor{yellow}
\multicolumn{2}{|l|}{Studentnaam: Joey Kaan \hspace{35pt} Studentnummer: 64808} \\
\hline
\rowcolor{yellow}
\multicolumn{2}{|l|}{Stagedocent: D. de Waard} \\ 
\hline
\rowcolor{yellow}
\multicolumn{2}{|p{\textwidth-1in}|}{Leerdoel/competentie: Versiebeheer volgens de Gitflow workflow gebruiken zodat er
altijd een fout vrije en uitrolbare versie van de applicatie beschikbaar is en als
er een fout optreedt deze snel weer ongedaan gemaakt kan worden door naar
een vorige versie terug te keren.} \\
\hline
\rowcolor{yellow}
\multicolumn{2}{|l|}{Cursus: Meewerkstage \hspace{35pt} Cursuscode: CU06322} \\
\hline
\rowcolor{yellow}
\multicolumn{2}{|l|}{Datum: \today} \\
\hline
\rowcolor{yellow}
\multicolumn{2}{|l|}{Titel en nummer van bewijs/bewijzen: } \\ [50pt]
\hline
\rowcolor{yellow}
\multicolumn{2}{|l|}{Oordeel: } \\
\hline
S & Geef voorbeelden van opdrachten (situaties) waarmee je kunt aantonen dat je de competentie hebt verworven. Beschrijf kort wat er aan de hand was of om welke opdracht het ging. 
\newline
\newline
De opdracht was Content2Connect. Deze was zo ver dat er een versie 1.0 kon worden uitgebracht. \\ 
\hline
\rowcolor{yellow}
T & Beschrijf de exacte rol/taak die jij had. Geef aan of het om een complexe taak ging en waaruit dat bleek. Wat moest jij doen?
\newline
\newline
Mijn rol was het klaarmaken van Content2Connect voor de release, Content2Connect te releasen en daarna deze ook te beheren. Met beheren wordt bedoeld of de release goed is gelukt, of alle functies ook werken op de productie server. Zo niet dan zal hier wat aanpassingen aan gedaan moeten worden. Hiernaast moet ook de feedback van klanten en van de interne gebruikers van ConnectSB worden verwerkt en bugs gefixed worden naarmate deze worden gevonden.
\\
\hline
A & Beschrijf de activiteiten die jij achtereenvolgens hebt ondernomen in het kader van deze opdracht. Wat heb je concreet gedaan?
\begin{itemize}
\item Functionaliteit voor Content2Connect afmaken.
\item Online zetten voor testen op test omgeving.
\item Test periode doorlopen waarbij getest wordt door management en community management.
\item Feedback verwerken van de test periode.
\item Online zetten op de productie omgeving.
\item Bugs gefixed die gaande weg ontdekt werden.
\item Nieuwe branch aangemaakt met de volgende naam: hotfix-testbugs.
\item Bugs opgelost in deze branch.
\item Hotfix-testbugs branch gemerged met master branch
\end{itemize}
\\
\hline
\rowcolor{yellow}
R & Beschrijf het resultaat van de opdracht en hoe de betrokken en er op reageerden. Wat is er vervolgens met dat resultaat gebeurd?
\newline
\newline
Het resultaat was een volledig werkende versie van  Content2Connect, de eerste dag hadden er al twee klanten gebruik gemaakt van het platform en dit was allemaal succesvol verlopen. De klanten waren heel enthousiast over het platform, het werkte heel prettig en het was vooral heel overzichtelijk. 
\\
\hline
R & Wat heb je ervan geleerd? Wat zou je volgende keer anders aanpakken en waarom?
\newline
\newline
Tijdens het klaarmaken van Content2Connect voor de release heb ik geleerd dat het continue testen heel erg belangrijk is. Bij ConnectSB is er nog maar sinds kort de regel dat er tests geschreven moeten worden en Content2Connect had deze tests nog niet. In Content2Connect worden er verschillende notificaties verstuurd met hierbij bijbehorende emails. De emails werden soms wel verzonden en soms niet bleek na de test periode. Als hiervoor tests geschreven waren had ik heel erg gemakkelijk kunnen zien waar de fout zat. Zo heb ik toen nog wel tests geschreven voor de email functie en hiermee kon ik dus heel gemakkelijk ook zien waar de fout zat.
\\
\hline
\rowcolor{yellow}
T & Geef een voorbeeld van een andere situatie waarin je deze competentie kunt toepassen
\newline
\newline
Deze competentie is eigenlijk te gebruiken gedurende de hele levensduur van de software. De Gitflow workflow gaat niet alleen over software die al in gebruik genomen wordt. Het is een workflow die het makkelijk maakt om aan aparte features te werken zonder dat de toestand van het development werk en de toestand van de code die productie gereed is wordt aangetast.
\\
\hline
\end{tabularx}